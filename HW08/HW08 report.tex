\documentclass{article}
\usepackage[utf8]{inputenc}
\usepackage{multicol}
\usepackage{listings}
\usepackage{verbatim}
\usepackage{color}
\usepackage{geometry}
\usepackage{float}
\usepackage{amsmath}

\usepackage{pdflscape}
\usepackage{hyperref}
\setlength{\belowcaptionskip}{-10pt}
\setlength{\abovecaptionskip}{-30pt}
\floatstyle{boxed} 
\restylefloat{figure}
\usepackage{graphicx}
\definecolor{codegreen}{rgb}{0,0.6,0}
\definecolor{codegray}{rgb}{0.5,0.5,0.5}
\definecolor{codepurple}{rgb}{0.58,0,0.82}
\definecolor{backcolour}{rgb}{0.95,0.95,0.92}

\lstdefinestyle{mystyle}{
	backgroundcolor=\color{backcolour},   
	commentstyle=\color{codegreen},
	keywordstyle=\color{blue},
	numberstyle=\tiny\color{codegray},
	stringstyle=\color{codepurple},
	basicstyle=\footnotesize,
	breakatwhitespace=false,         
	breaklines=true,                 
	captionpos=b,                    
	keepspaces=true,                 
	numbers=left,                    
	numbersep=5pt,                  
	showspaces=false,                
	showstringspaces=false,
	showtabs=false,                  
	tabsize=2
}

\lstset{style=mystyle}
\title{Data Mining\\
		Home work 08\\Machine Learning Start... }
\author{Aqeel Labash\\ \textbf{Lecturer:} Jaak Vilo}
\date{29 March 2016}

\geometry{
	a4paper,
	total={170mm,257mm},	
	left=10mm,
	top=5mm,
}
\begin{document}
	\maketitle
\section*{First Question}
The quality of classifier as I understood from the page is when we can classify accurately depending on that classifier.By that I mean to have a certain point where it completely separate data into two groups.\\ Another meaning for quality of classifier might be if the classifier really represent a real case or just something happened with high probability in training dataset.If it's just high probability then depending on that classifier will just make our model worst at predicting with real data or unseen data.\\
\section*{Second Question}
\section*{Third Question}
\section*{Fourth Question}
\section*{Fifth Question}
\section*{Sixth Question}
\end{document}